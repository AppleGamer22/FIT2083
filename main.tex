\documentclass[11pt, a4paper, fleqn]{article}
\setlength\parindent{0pt}
\usepackage{design_ASC}
\usepackage{amssymb}
\usepackage{array}
\usepackage{varwidth}
\usepackage{cancel}
\usepackage{ulem}
\usepackage{amssymb, amsmath, bm}
\title{FIT2004 Studio \#3}
\date{\today}
\author{Omri Bornstein | 31570895}
\begin{document}
    \maketitle
    \begin{enumerate}
        \item Write psuedocode for insertion sort, except instead of sorting the elements into non-decreasing order, sort them into non-increasing order. Identify a useful invariant of this algorithm.
        \begin{itemize}
            \begin{lstlisting}
                function INSERTION_SORT(A[1..n])
                    i = 2
                    while i <= n do
                        key = A[i]
                        j = i - 1
                        while j >= 0 and key > A[j]:
                            A[j + 1] = A[j]
                            j = j - 1
                        end while
                        i = i + 1
                    end while
                end function
            \end{lstlisting}
            \item All items at the interval $A[1...i-1]$ are sorted.
        \end{itemize}
        \item Consider the following algorithm that returns the number of occurrences of target in the sequence A. Identify a useful invariant that is true at the beginning of each iteration of the while loop. Prove that it holds, and use it to prove that the algorithm is correct.
        \begin{itemize}
            \begin{lstlisting}
                function COUNT(A[1..n], target)
                    count = 0
                    i = 1
                    while i <= n do
                        if A[i] = target then
                            count = count + 1
                        end if
                        i = i + 1
                    end while
                    return count
                end function
            \end{lstlisting}
            \item The variable count stores the number of times the input target appears in the interval $A[1...i]$.
            \item The invariant described above holds because the if-statement detects if $A[i]$ has the same value as the input target and then updates the variable and output count.
            \item Since the algorithm increments the variable count by 1 for every items that has the same value as the input target, the  algorithm is correct.
        \end{itemize}
    \end{enumerate}
\end{document}